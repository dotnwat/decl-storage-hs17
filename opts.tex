\section{Discussion and Conclusion}
\label{sec:opts}

It's clear that storage systems are currently in the midst of significant
change and there are few guide posts available to developers navigating a
large and complex design space. This has served as the primary source
of motivation for our use of declarative languages. And while our
implementation does not yet map a declarative specification on to a particular
physical design, the specification provides a powerful infrastructure for
automating this mapping and achieving other optimizations.
Given the declarative nature of the interfaces we have defined, we can draw
parallels between the physical design challenges described in this paper and
the large body of mature work in query planning and optimization.  The Bloom
language that we use as a basis for a declarative specification
produces a dataflow graph that can be used in static analysis, and we
envision that this graph will be made fully available to the storage system to
exploit before and during runtime.

We are currently considering the scope of optimizations that are possible with
such a dataflow model in the context of the storage systems. For instance, without
semantic knowledge of an interface, batching techniques described in
Section~\ref{sec:batch} are limited to optimizations such as selecting magic
values for timers and buffer sizes. Semantic information expands the design
space, permitted intelligent reordering or coalescing that depends on
relationships between operations, going beyond what auto-tuning has
previously considered.

{\bf Conclusion.} Optimizing every new or changed application as storage
systems evolve is obviously impractical.  A storage system is not the same as a
database system, but techniques from database optimization can potentially be
leveraged to address complexity, performance and transparent portability for
applications running on evolving storage systems.  Generalizing from the
example we described, we think this approach is innovative and promising. 

%{\bf Conclusion.} With each new application-specific storage interface one
%must consider that the logical conclusion is a future in which little to no
%distinction exists between storage and database systems; or not. But on the
%road to either future intermediate solutions like the one we have proposed are
%necessary to manage complexity and portability.
%Looking beyond standard forms of optimization decisions that seek to select
%an appropriate mix of low-level I/O interfaces, data structure selection is an
%important point of optimization. For instance in Section~\ref{sec:dspace} we
%showed that using the bytestream for metadata management as opposed to the
%key-value interface offered superior performance. However the unstructured
%nature of the bytestream data model imposes no restrictions on implementation
%or storage layout. Integration of common indexing techniques into an optimizer
%combined with a performance model will allow our CORFU interface to derive
%similar optimizations when appropriate. Similar degrees of freedom can be
%imagined when handling other approaches to implementing the CORFU sequencer
%service. Given its soft-state nature heavy-weight processes that enforce
%durability can be circumvented in favor of shorter code paths that optimize
%for throughput and latency.
%
%
%For
%example today object classes are represented as black boxes from the point of
%view of the OSD execution engine.  Understanding the behavior of an object
%class may allow intelligent prefetching. Another type of analysis that may be
%useful for optimization is optimistic execution combined with branch
%prediction where frequent paths through a dataflow are handled optimistically.
